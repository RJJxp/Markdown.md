\chapter{陈晨个人实习报告}

如何将理论与实际联系,进行知识向技能的改变,这是我一直在思考的问题,如果只能纸上谈兵,知识可能就会失去活力。我们在上次暑假期间进行的测量就是在提高我们的仪器操作水平和增强我们实地进行测量的能力,而不是让我们仅仅掌握零碎的知识点,这次的摄影测量实习我也有同样的感受,实习是一个整体,让我掌握了从书本中无法习的得流畅的知识体系,让我感受深刻。

\section{实习感想}

在这次摄影测量中,我们结合上学期所学习的摄影测量的知识,将摄影测量的工作流程完整的体验了一遍,在外业进行调绘,在内业进行点之记编写、数字成图等工作,着重进行了相对定向和绝对定向的编程。

有了这一次经历,我更深刻地感受到“纸上得来终觉浅,绝知此事要躬行”,不仅是因为一次次调试程序带来的麻烦;而且是因为学习的知识只有在被应用的时才真正知道自己有没有掌握,应用知识才是学习知识的第一驱动力(印象最深的就是编写相对定向程序的过程,下面会详细说明)。


\subsection{相片联测和同名点坐标量测}
首先这次实验从实地测量开始,选点、观测与记录都是我们小组成员共同完成。我们按照任务书上的指导,每个点测了两遍作为校核。与课本中提到的不同,在现实里我们要考虑更多的东西,比如树木的遮挡,建筑是否有变化,如果周围的控制点不够或者控制点较远我们如何解决,也体现了我们对知识的掌握程度。更具体的例子就是,在第一次测量时候,由于我们对无棱镜测量的特点没有充分了解,选取了一个有较多树枝遮挡的建筑角点,这就造成了偏差较大。后来我们及时意识到问题,并更换了点。

接着组长分配任务,按照摄影测量实习要求,我们组的每个人都在影像上找6对同名点,作为相对定向的基础。在找点的时候,要特别小心仔细,要注意时候图片不能缩放的过大,避免找不到点。这是一个基本功,在后面的CAD成图中也涉及到用PS找像素坐标。

\subsection{摄影测量编程}
编程是一个非常重要的环节,当然掌握知识是必要的前提条件。

我觉得这次编程对于我来说是意义重大的。其一是知识的重温,虽然在上学期我学习了相对-绝对定向是将像素坐标转化为物方坐标的一种基本方法,也了解它的基本思想,相对-绝对是先求解左右相片的相对位置,然后再逐步求解其在空间中的位置,但我却不是特别清楚具体操作的流程,我可以很流畅的把流程图背下来,但是这些知识没有内化,在实习中我实现了知识的内化吸收。其二是克服了编程的畏难情绪,面对这样一个大的问题,在一开始我没有分解的意识,我对于把这些流程转化为准确高效的代码很惶恐,但如果不开始就什么都完不成,一步一步,即使很痛苦的Debug也是通往结果的一个过程吧。后来想清楚之后,感觉学习就应该逼迫自己快点经历,有过经历之后,就不再对结果迷茫,结果就不那么不可知。其三是技能的掌握,程序的编写也让我复习了 MATLAB 的使用,这是一项非常综合的锻炼,让我得到很大的提升。

由于在上学期我们就已经进行了内定向的编程,所以我就直接从相对定向开始编写。正如上文提到的,在单独法相对定向时的知识储备足够了,但是有很多心态上的不足,造成效率比较低下。单独法相对定向就是在利用我们找到的同名点的像素坐标求出的框标坐标之后,求出五个相对定向元素。不过照着流程图我也很快完成了第一次相对定向,求出了相对定向元素,与小组成员比较之后基本满足精度要求。在这里我们有一个不能忽略的问题:这些点是否符合要求吗(即这些点在两张相片上真的是同名点吗)?

为了解决这个问题,我在运用求得的独立像对相对定向参数以及量测的同名点坐标,根据前方交会计算式计算模型坐标,得到像片控制点、待定点的模型坐标之前,我们通过前方交会计算这些点的上下视差来检验点的质量,删除上下视差超过限度的点,重新求取相对定向元素。经过前方交会的过程我们发现有一些点中有的上下视差过大,这些点就是我们要剔除的点。当剔除的阈值(上下视差)选择±0.3和±0.5时,发现相差的结果并不大。后来,为了小组内部成员能够相互检查督促,我们选取了相同的点进行相对定向。这个问题是我们在上学期中完全不会遇到的,因为我们默认这些点肯定是正确的,但是实际生产中如何肯定要进行点的质量检测,这就是学习和实习的不同。

在后续的模型点坐标计算和绝对定向参数的求解中,我遇到了很多问题,不过在经过思考和讨论后,问题都得到了合理的解决。问题列表如下:
\begin{itemize}
	\item[-] 绝对定向中Bx的确定\\
	Bx代表摄影基线的长度。相对定向中,bx其实是一个无关紧要的量,因此,我算出来相对定向的值是正确的;然而到了绝对定向,出现问题了,七参数中z异常大!然而找不到问题所在,最后在调试时,发现不对,才改正过来。
	\item[-]参数过多,传值频繁。\\
	在数值传递过程中就会出现变量混乱,变量名污染的情况。为了避免这种状态我将不同功能用不同的函数分开求解,才解决这个问题。
	\item[-]数据录入错误/公式录入错误。\\
	在绝对定向输入系数阵时,我把一个项的系数阵的正负号搞错了,这也是绝对定向不收敛的原因之一。这可能是最低级的错误,但是这次我又犯了。
	\item[-]点的质量。\\
	同上文一样,在进行绝对定向过程中要对控制点的质量进行检查,我们的做法是剔除上下视差Q过大的点,留下最好的11个点作为计算绝对定向元素的点,剩下的较好的点做为检查点。
\end{itemize}

\subsection{校园调绘}

在调绘的过程中,我们一边数楼层一边调查这些楼是什么结构,分了两天才将全校调绘完成。调绘完成后,我才知道同济的校园是这么大,这些都是我之前没有认真体会的。而且我们还发现有着一百多年历史的楼-文远楼,这些都是我不曾注意的。可以说这次调绘增强了我对我们学校的了解,也更深刻地体会到我们学校丰富的文化底蕴。

\subsection{三维建模}
我们尝试对一位坐在椅子上的同学进行扫描建模,相对与椅子的模型,人体建模结果较差,整体较模糊。特别是脸部特征,基本全部没有。分析其原因,可能是人体皮肤对红外线反射较少,另同学配戴眼镜,也会有一定反射。另外镜头在移动的时候可能没有控制好速度,这也是原因之一。通过实验,我们熟悉了软件使用,了解了三维建模基本过程。

\subsection{总结}
这次实习中,我们小组里精诚合作,相互帮助,增强了我们的团结协作能力,培养了同学之间深厚的情谊,让我们在相互学习中进步。

虽然这次做的成果并不是很完美,但通过对实习成果的分析,我了解了摄影测量作业的基本流程,对摄影测量课程有了更深更具体的体会。在这次实习中, 老师不再是单纯的给我们讲解知识, 而是给了我们大量的时间去实践,再针对问题进行解答,很好的锻炼了我们的各种能力,既让我们更加清晰的认识了摄影测量学这门学科, 使课本知识得到了巩固,也让编程能力进行了历练,使我获益良多。

\section{程序及计算结果}

\subsection{内定向}
\textbf{内定向参数:}
\begin{equation}
\begin{array}{lll}
h_0=-46.080000 & h_1=0.012000 & h_2=0.000000 \\
k_0=82.944000 & k_1=-0.000000 & k_2=-0.012000
\end{array}
\end{equation}

\textbf{内定向程序:}
\begin{lstlisting}[caption=orientation.m文件]
function orientation()%内定向
format long
[a1,a2,a3,a4]=textread('camera.txt','%f%f%f%f');%处理过的框标的理论位置
a=[a1 a2 a3 a4];
theoreticalpositions = a(:,1:2);%框标的理论位置
PC = a(:,3:4);%框标像素坐标
%x=a0+a1*x'+a2*y';y=b0+b1*x'+b2*y'
L=theoreticalpositions';
L=L(:);
B=zeros(8,6);
for i=1:2:8
B(i,1)=1;
end
for i=2:2:8
B(i,4)=1;
end
for i=1:2:8
B(i,2)=PC(((i+1)/2),1);
end
for i=1:2:7
B(i,3)=PC(((i+1)/2),2);
end
for i=2:2:8
B(i,5)=PC((i/2),1);
end
for i=2:2:8
B(i,6)=PC((i/2),2);
end
X=inv(B'*B)*(B')*L;
V=-B*X+L;
omega=sqrt(V'*V)/2;
fid=fopen('ndx.txt','wt');
fprintf(fid,'%s\n','Internal fixed parameter(h0,h1,h2,k0,k1,k2) are shown as follows:');
fprintf(fid,'%s\n','precision:');
fprintf(fid,'%f\n',omega);
[row,col]=size(X);
for i=1:1:row
for j=1:1:col
if(j==col)
fprintf(fid,'%f\n',X(i,j));
else
fprintf(fid,'%f\n',X(i,j));
end
end
end
\end{lstlisting}

模型:
\begin{equation}
\begin{bmatrix}
x \\ y
\end{bmatrix}
=\begin{bmatrix}
h_1 & h_2 \\
k_1 & k_2 
\end{bmatrix}
\begin{bmatrix}
i \\ j
\end{bmatrix}
+\begin{bmatrix}
h_0 \\ k_0
\end{bmatrix}
\end{equation}

\subsection{相对定向}
\textbf{相对定向参数:}
\begin{equation}
\begin{array}{lll}
\phi_1=-0.012473 & & \kappa_1=0.028427 \\
\phi_2=0.013329 & \omega_2=-0.022502 & \kappa_2=0.036377
\end{array}
\end{equation}

\textbf{相对定向程序:}
\begin{lstlisting}[caption=ca\_faducialmarks.m文件]
function ca_faducialmarks()%计算框标
[a1, a2, a3, a4]=textread('C:\Users\c\Desktop\新建文件夹\新建文本文档.txt','%f %f %f %f');
num=length(a1);
photo1=[ones(num,1) a1 a2];
photo2=[ones(num,1) a3 a4];
orientation1=textread('ndx.txt','%n','headerlines',3);
orientation2=textread('ndx.txt','%n','headerlines',3);
h1=orientation1(1:1:3,:);k1=orientation1(4:1:6,:);
h2=orientation2(1:1:3,:);k2=orientation2(4:1:6,:);
psx1=photo1*h1;
psy1=photo1*k1;
f=120.000;
psx2=photo2*h2;
psy2=photo2*k2;
fid=fopen('kuangbiao.txt','wt');
[row,~]=size(psx1);
for i=1:1:row
fprintf(fid,'%f %f %f %f\n',psx1(i,1),psy1(i,1),psx2(i,1),psy2(i,1));
end
st=fclose(fid);
return;
\end{lstlisting}

\begin{lstlisting}[caption=ca\_R.m文件]
function R=ca_R(w,p,k)
Rz=[cos(k) -sin(k) 0;sin(k) cos(k) 0;0 0 1];
Ry=[cos(p) 0 -sin(p);0 1 0;sin(p) 0 cos(p)];
Rx=[1 0 0;0 cos(w) -sin(w);0 sin(w) cos(w)];
R=Ry*Rx*Rz;
\end{lstlisting}

\begin{lstlisting}[caption=relative\_oriention.m文件]
function relative_oriention()%相对定向
format long
f=120.000;%mm
[x1,y1,x2,y2]=textread('kuangbiao.txt','%f%f%f%f');
w1=0;
%Number of points of the same name
dotnum= length(x1); 
%initial relative parameters :
p1=0;k1=0;w2=0;  p2=0;  k2=0; 
z=-f*ones(1,dotnum);
X0=[p1;k1;p2;w2;k2];
deltax=ones(5,1);
ct=0;
while (abs(deltax(1))>0.0003 || abs(deltax(2))>0.0003 || abs(deltax(3))>0.0003 || abs(deltax(4))>0.0003 || abs(deltax(5))>0.0003)
%计算像点的像空间辅助坐标X1, Y1, Z1, X2, Y2, Z2
ct=ct+1;
%rotational matrix R
R2 = ca_R(w2,p2,k2);
R1 = ca_R(w1,p1,k1);
P1=R1*[x1';y1';z];
P2=R2*[x2';y2';z];
%Generate error equation
A=zeros(dotnum:5);
A(:,1)=(-P1(1,:).*P2(2,:)./P1(3,:))';
A(:,2)=P1(1,:)';
A(:,3)=(P2(1,:).*P1(2,:)./P1(3,:))';
A(:,4)=(P1(3,:)+P1(2,:).*P2(2,:)./P1(3,:))';
%A(:,4)=-(P1(3,:)-P1(2,:).*P2(2,:)./P1(3,:))';
A(:,5)=-P2(1,:)';
L=f*(P1(2,:)./P1(3,:)-P2(2,:)./P2(3,:))';
deltax=inv(A'*A)*A'*L;
p11=p1+deltax(1,1);
k11=k1+deltax(2,1);
p21=p2+deltax(3,1);
w21=w2+deltax(4,1);
k21=k2+deltax(5,1);
% p1;k1;p2;w2;k2
if ct>6
msgbox('Maybe you are wrong.Please check it.','Hello:)','warn');
break;
end;
p1=p11;
k1=k11;
p2=p21;
w2=w21;
k2=k21;
V= - (A * deltax);
sigma = sqrt(V' * V);
exx = sigma * inv(A' * A);
variances = diag(exx);
deviations = sqrt(variances);
end
fid=fopen('ro_conclusion.txt','wt');
c=fprintf(fid,'%s\n','The relative orientation parameter are');
c=fprintf(fid,'%f\n',p1);
c=fprintf(fid,'%f\n',k1);
c=fprintf(fid,'%f\n',p2);
c=fprintf(fid,'%f\n',w2);
c=fprintf(fid,'%f\n',k2);
st=fclose(fid);
return;
\end{lstlisting}

\subsection{绝对定向}
\textbf{绝对定向参数:}\\
$\phi=0.011271124539547$  \\
$\omega=0.005631855652775$\\  
$\kappa=0.000538186734723$ \\
$\lambda=1.000708892036554$ \\
dx=0,\quad dy=0,\quad dz=0  

 \textbf{地面点的重心化作标:} 
 5490.944272727273\quad 3057.336272727273\quad 008.125272727273

\textbf{绝对定向程序:}
\begin{lstlisting}[caption=absolute\_orientation.m文件]
function absolute_orientation()
[xp,yp,zp,Q]=textread('3.txt','%f%f%f%f');
[N,E,H]=textread('2.txt','%f%f%f');
Xs=0;
Ys=0;
Zs=3;
p=0;
w=0;
k=0;
lamada=1;
M=7680*0.2/46.08/2;%M=1/((46.0800*2)/(7680*0.2));
pointnum=length(xp);
P=[xp,yp,zp];
TP=[N,E,H];%地面摄测坐标
PG=sum(P)/pointnum;
TG=sum(TP)/pointnum;
sumscale=0;
for i=1:1:pointnum   
Pg(i,1)=P(i,1)-PG(1);
Pg(i,2)=P(i,2)-PG(2);
Pg(i,3)=P(i,3)-PG(3);%(1);
Tg(i,1)=TP(i,1)-TG(1);
Tg(i,2)=TP(i,2)-TG(2);
Tg(i,3)=TP(i,3)-TG(3);
JL(i)=sqrt(Pg(i,1)*Pg(i,1)+Pg(i,2)*Pg(i,2)+Pg(i,3)*Pg(i,3));
jl(i)=sqrt(Tg(i,1)*Tg(i,1)+Tg(i,2)*Tg(i,2)+Tg(i,3)*Tg(i,3));
scale(i)=jl(i)/JL(i);
sumscale=sumscale+scale(i);
end
meansacle=sumscale/pointnum;
for i=1:1:pointnum   

Pg(i,1)=Pg(i,1)*meansacle;
Pg(i,2)=Pg(i,2)*meansacle;
Pg(i,3)=Pg(i,3)*meansacle;
end
ct=0;dw=1;dp=1;dk=1;
while(abs(dw)>1e-5|| abs(dp)>1e-5||abs(dk)>1e-5)
R=ca_R(w, p, k);  %求解余弦函数  
for i=1:1:pointnum %组成法方程系数阵%求常数项
A(3*i-2,:)=[1, 0, 0, Pg(i,1), -Pg(i,3), 0, -Pg(i,2)];
A(3*i-1,:)=[0, 1 ,0 ,Pg(i,2),0, -Pg(i,3) , Pg(i,1)];
A(3*i,:)=[0 ,0 ,1 ,Pg(i,3), Pg(i,1) ,Pg(i,2), 0];
l(i,:)=Tg(i,:)-lamada*Pg(i,:)*R'-[Xs Ys Zs];
L(3*i-2)=l(i,1);
L(3*i-1)=l(i,2);
L(3*i)=l(i,3);
end
X=inv((A')*A)*(A')*L'; %求解改正数X
p=p+X(5,1);
w=w+X(6,1);
k=k+X(7,1);
lamada=lamada*(X(4,1)+1);
Xs=Xs+X(1,1);
Ys=Ys+X(2,1);
Zs=Zs+X(3,1);
dp=X(5,1);
dw=X(6,1);
dk=X(7,1);
V=A*X-L;
Qx=inv((A')*A);
m=sqrt(V'*V/(3*pointnum-7));
mx=m*sqrt(Qx(1,1));
my=m*sqrt(Qx(2,2));
mz=m*sqrt(Qx(3,3));
mr=m*sqrt(Qx(4,4));
mq=m*sqrt(Qx(5,5));
mw=m*sqrt(Qx(5,5));
mk=m*sqrt(Qx(7,7));
Accuracy=[m,mx,my,mz,mr,mq,mw,mk];
jP=[Xs,Ys,Zs,p,w,k,lamada];
ct=ct+1;
end
\end{lstlisting}

以下程序用于检查绝对定向元素:
\begin{lstlisting}[caption=inspect.m文件]
for i=1:n
XYZ(i)=lamada*Pg(i,:)*R'+[X(1,1) X(2,1) X(3,1)];%PG为待求点的重心化地面摄影测量坐标
XYZQ(i)=XYZ(i)+PG;
xc(i)=XYZQ(i,1)-XYZ(i,1);
yc(i)=XYZQ(i,2)-XYZ(i,2);
zc(i)=XYZQ(i,3)-XYZ(i,3);
wuchashi(i)=sqrt(xc(i)*xc(i)+yc(i)*yc(i)+zc(i)*zc(i));
end
xzwc=sqrt(xc*xc'/(n-1));
yzwc=sqrt(yc*yc'/(n-1));
zzwc=sqrt(zc*zc'/(n-1));
\end{lstlisting}

\section{调绘成果}
%这个待会才能完成

\begin{figure}[htbp]
\centering
\caption{调绘图概览}
\includegraphics[width=\textwidth]{diaohuicc.jpg}
\end{figure}

\begin{figure}[htbp]
\centering
\caption{调绘图细节}
\includegraphics[width=\textwidth]{diaohui1cc.JPG}
\end{figure}


\section{同名点}

\begin{table}[htbp]
	\centering
	\begin{tabular}{lrrrr}
		\toprule
		& \multicolumn{2}{c}{63} & \multicolumn{2}{c}{64} \\ \hline
		点号 & \multicolumn{1}{c}{i} & \multicolumn{1}{c}{j} & \multicolumn{1}{c}{i} & \multicolumn{1}{c}{j} \\ \midrule
	    26 & 4933 & 2848 & 1773 & 2469 \\
      27 & 5085 & 2922 & 1929 & 2547 \\
      28 & 5252 & 3003 & 2097 & 2633 \\
      29 & 5541 & 3095 & 2375 & 2732 \\
      30 & 5215 & 3141 & 2061 & 2772 \\
      31 & 5670 & 3169 & 2501 & 2809 \\ \bottomrule
	\end{tabular}
	\end{table}