\chapter{实习各项目具体情况报告}

\section{航测外业实习}

航测外业实习包括:像片联测、航片判读调绘。

\subsection{像片联测}

\label{sub:waiye}
利用摄影测量方法进行地形图测绘有几种方法,但不论采用哪种方法都需要测定一些像点的对应的地面坐标,这些测定了地面坐标的像点称为\textbf{像片控制点}。将像片上的点根据大地点及水准点测得其平面位置和高程的过程称为\textbf{像片联测}。

摄影测量与遥感实习的像片联测工作包括像控点坐标的量测与点之记的制作。实习中量测的像控点将用于后面航测内业实习中的绝对定向以及同济航空影像的纠正。选取像片控制点的原则如下:

\begin{enumerate}[label=\alph*.]
    \item 控制点必须在影像上能清晰辨认;
    \item 实地能找到与影像一致的控制点;
    \item 该点易于测量。
\end{enumerate}

像片联测实习在同济大学校本部范围内进行,测量所选坐标系是同济独立坐标系。实际测量过程使用SET530R索佳全站仪进行。该全站仪可以无棱镜测量,无棱镜测程为150米(或采用其他类似类型的全站仪进行)。测量工作利用校园内布置的导线网作为控制施测。像控点量测步骤及方法如下:

\begin{enumerate}
\item \textbf{勘探选点}\\
从像片上选取同济大学校园内若干个(不少于6个)可量测特征点作为像片控制点,要求分布大致均匀,能在像片上清晰判断,实地便于测量。并到实地踏勘确认。

\item \textbf{坐标量测}\\
利用同济校园内布设的导线网和高程网作为控制基础,利用电子全站仪测量所选像控点的坐标$(X,Y,Z)$。测量可采用普通测量中所学的各种方法(如支导线法、前方交会等方法),观测需量测的像片控制点,并将测量结果记录下来$(X,Y,Z)$;对每一像控点测量都应有检核方法,得到该像控点的两次量测结果$(X_1,Y_1,Z_1),(X_2,Y_2,Z_2)$。

\item \textbf{测量精度的要求}\\
为提高测量精度,要求每个待定控制点的两站量测值之差满足:$dX<0.02\si{m},
dY<0.02\si{\metre},dZ<0.1\si{\metre}$的要求。由于校园内建筑物较为密集,互相通视的点很少能达到三个,在测量中还需利用全站仪后视定向之后直接打后视点的坐标来进行检查,该值与所给的已知值的误差,$X(N)$方向为$1\sim 2\si{\milli\metre}$,$Y(E)$方向为$1\sim 2\si{cm}$,$Z$(高程)方向的误差小于$3\si{cm}$。最后取两次观测结果的平均值作为最终像控点的坐标量测成果。

在像控点量测结束后应进行点之记的制作,以利于后续处理人员在航片影像上对所测控制点的位置能确认无误。由于采用的是航片数字影像,因此点之记的制作过程如下:
\begin{enumerate}[label=\Alph*.]
\item 在所给航片影像上判断所测像控点的位置,并将该点的像素坐标记录下来---$(i,j)$;
\item 以该点为中心,在原影像上取一100px$\times$100px的方块,其中心点即所测量的纠正控制点位置,将该图像存为点之记图片文件,文件以该像控点的点号命名;
\item 对该点位位置进行简要的说明;
\item 生成点之记文件。
\end{enumerate}
\end{enumerate}

\subsection{像片判读调绘实习}

像片判读是根据航摄像片上地物影像的特征来识别地物的实质内容。像片判读分野外判读、室内判读和综合判读。在野外实地对像片上影像进行的像片判读即为野外判读。一般将两者结合起来,用野外判读和实验方法先取得标准判读像片,然后在室内以标准判读像片为准绳与其他航摄像片相比照,这样可以节省大量野外工作。像片调绘是在像片判读的基础上实地检查室内判读的正确性,并在影像上进行标注;对影像上没有的地物通过补测与调绘后绘注于影像上。

\noindent 实习中调绘范围:同济大学本部校园、南校区及附近地区;\\
调绘影像的航摄时间:2005年,\\
影像分辨率:约$0.2\si{m/px}$

实习中调绘判读的影像主要是同济校本部,调绘判读的内容包括:
\begin{enumerate}
\item 建筑物的名称、层数、结构(包括砖石、砖木和钢混结构)、楼内分区单位及用途(商用、教学、居住、办公);
\item 道路名称、宽度,其中宽度需用皮尺到实地量测;
\item 绿化(细分为林地和草地);
\item 土地用途(包括教学、办公、商住、住宿四类)。
\end{enumerate}

此外,还将同济校园范围在影像上标注出来。此次实习中调绘的目的主要是确定校园中地物名称等属性,为内业成图提供属性信息。

根据所给航片影像到实地进行判读调绘,\textbf{要求}:走到、看到、问到。

调绘结果利用通用图像处理软件photoshop软件,将实地调查所得资料标注在校园数字影像上。\textbf{要求}:注记文字的位置要能明确表明所指地物;有变化的地物要予以圈出,并标出当前的名称及其属性。用photoshop软件将调查成果注记在图上,生成一幅调绘图像。


\section{航测内业实习}
\label{sec:neiye}
航测内业实习包括:同名点坐标量测(相对定向点、像片控制点、待成图地物点),编程(包括数字内定向、解析相对定向、解析绝对定向、前方交会),部分地物的CAD成图以及航片影像的数字微分处理。

\subsection{同名点坐标量测}
\label{sub:tongming}

同名点坐标量测包括相对定向点、像片控制点、待定地物点的量测,在本实习中,利用photoshop软件打开左右航片影像,人工判断左右片的同名点并将同名点对的行、列号记录下来,生成同名点对文件。

% (如表\ref{tab:23})。

% \begin{table}[htbp]
% \centering
% \begin{tabular}{ccccc}
% \toprule
% ID & \multicolumn{2}{c}{左片(119058)} & \multicolumn{2}{c}{右片(119057)} \\
% \midrule
%  & i & j & i & j \\
% $x_1$ & 7668 & 86 & 4651 & 463 \\
% $x_2$ & 6198 & 195 & 3205 & 562 \\
% $x_3$ & 3683 & 346 & 699 & 702 \\
% $x_4$ & 6914 & 2560 & 3899 & 2909 \\
% \bottomrule
% \end{tabular}
% \caption{左右片坐标}
% \label{tab:23}
% \end{table}

\subsubsection{相对定向点的量测}

像对的相对定向至少需要五个同名点,但是在模拟与解析测图作业中常常利用六点法或者九点法选点。六点法为:围绕两张像片的重叠区域四周上下两边上左右各选一点,中间一条线的两端各选一点,一共六点;九点法为:上下两边和中间一条线的左中右各选一点,一共九点。但是在数字摄影测量中相对定向数目大大超过上面的点数。

实习中每人完成5点的量测,为了避免量测的点集中在一个区域,在量测中可将航片由上到下分为六条带,每条上量测4-6点,总共量测了30-40点,生成相对定向点文件,用于参与后面的相对定向的平差解算。

\subsubsection{像片控制点的像点坐标量测}

利用像控点点之记图像,文字注记以及在所给航片影像上的像素位置,判断该像控点在其他航片影像(左片或右片)上的位置将该点的像素坐标记录下来---$(i,j)$,生成此像控点的坐标文件,用于前方交会计算该像控点的模型坐标。

\subsubsection{待成图地物点的像点坐标量测}

对于待定地物点,采用同样的方法,量测该地物特征点(如房屋角点、道路拐弯点等)在左右航片上的像素位置,并记录下来。注意:在记录这些地物点时,应给其编号、按照一定的顺序记录,并绘出点位连接的草图,以便于后续的CAD成图(类似于普通测量中的数字平板测图记录的要求)。

\subsection{解析摄影测量编程练习}
\label{sub:biancheng}

\begin{figure}[htbp]
\centering
\caption{流程图}
\includegraphics[width=\textwidth]{flow.PNG}
\label{fig:22}
\end{figure}

实习中解析摄影测量编程包括数字内定向,解析相对定向,前方交会,解析绝对定向,最终实现相对定向---绝对定向解析摄影测量的整个过程,主要流程如图\ref{fig:22}所示。编程可采用同学已学的编程工具如VB、matlab、VC等。



\subsubsection{数字内定向}

\begin{enumerate}
\item \textbf{内定向参数求取}\\
方法:利用航摄像片上的四个框标点的理论位置以及四个框标点的像素坐标为依据,通过最小二乘法计算内定向参数。\\
计算式:
\begin{equation}
\begin{bmatrix}
x \\ y
\end{bmatrix}
=\begin{bmatrix}
h_1 & h_2 \\
k_1 & k_2 
\end{bmatrix}
\begin{bmatrix}
i \\ j
\end{bmatrix}
+\begin{bmatrix}
h_0 \\ k_0
\end{bmatrix}
\label{eq:convert}
\end{equation}
参数:$h_0,h_1,h_2,k_0,k_1,k_2$ \\
求解过程:列误差方程$\rightarrow$ 法化 $\rightarrow$ 求参数
\begin{solution}
将式\ref{eq:convert}列为关于参数的形式:
\begin{equation}
\begin{bmatrix}
x \\ y 
\end{bmatrix}
=
\begin{bmatrix}
1 & i & j & 0 & 0 & 0\\
0 & 0 & 0 & 1 & i & j
\end{bmatrix}
\begin{bmatrix}
h_0 \\ h_1 \\ h_2 \\ k_0 \\ k_1 \\ k_2
\end{bmatrix}
+\Vvarep
\label{eq:erroreq}
\end{equation}

将4个点的框标坐标$(x_1,y_1),\cdots,(x_4,y_4)$和像素坐标$(i_1,j_1,\cdots,i_4,j_4)$带入可列出8个方程:
\begin{equation}
\begin{bmatrix}
x_1 \\ y_1 \\ \vdots \\ x_4 \\ y_4 
\end{bmatrix}
=
\begin{bmatrix}
1 & i_1 & j_1 & 0 & 0 & 0\\
0 & 0 & 0 & 1 & i_1 & j_1 \\
\vdots & & \vdots & & & \vdots \\
1 & i_4 & j_4 & 0 & 0 & 0\\
0 & 0 & 0 & 1 & i_4 & j_4 
\end{bmatrix}
\begin{bmatrix}
h_0 \\ h_1 \\ h_2 \\ k_0 \\ k_1 \\ k_2
\end{bmatrix}
+\Vvarep 
\end{equation}
记为:
\begin{equation}
\Vy=\MX\Vbeta+\Vvarep
\end{equation}
参数的最小二乘估计为
\begin{equation}
\hat\Vbeta=(\MX\myt\MX)^{-1}\MX\myt\Vy
\end{equation}
\end{solution}
\item \textbf{将像点扫描坐标转化为框标坐标}\\
用参数的最小二乘估计代替公式\ref{eq:erroreq}中的参数向量可将点P的像素坐标$(i,j)$转换到框标坐标$(x,y)$:
\begin{equation}
\hat\Vy_p=\MX_p\hat\Vbeta
\end{equation}
\end{enumerate}

\subsubsection{单独相对相对定向编程}

\noindent\textbf{目的}:编写单独像对相对定向程序,利用\ref{sub:tongming}中量测的相对定向同名点解算两像片之间的相对定向元素并进一步利用前方交会公式计算模型点坐标,为后续的绝对定向、待定地物点的坐标计算做准备。\\
\textbf{地点}:机房、宿舍。\\
\textbf{使用仪器与设备}:电脑。 \\
\textbf{相对定向原理与步骤}:根据同名射线相交位于同一个平面上,利用共面条件方程,以及量测的同名像点坐标解求五个相对定向元素。

单独像对相对定向是以摄影基线作为像空间辅助坐标系的$X$轴,以左摄影中心$S_1$为原点,左像片主光轴与摄影基线$B$组成的主核面为$XZ$平面,构成右手直角坐标系$S_1-X_1Y_1Z_1$。此时,左、右像片的相对方位元素分别为:
\begin{description}
\item[左片] $Xs_1=0,Ys_1=0,Zs_1=0,\phi_1,\omega_1=0,\kappa_1$
\item[右片] $Xs_2=b,Ys_2=0,Zs_2=0,\phi_2,\omega_2,\kappa_2$
\end{description}
需要求解的相对定向元素为:左片$\phi_1,\kappa_1$,右片$\phi_2,\omega_2,\kappa_2$。

在介绍解算算法之前我们先介绍一下Gauss-Newton法。它的基本思想是使用泰勒级数展开式去近似地代替非线性回归模型,然后通过多次迭代,多次修正回归系数,使回归系数不断逼近非线性回归模型的最佳回归系数,最后使原模型的残差平方和达到最小。\footnote{这个算法的停车条件是驻点条件,或者用改正数条件$||x^{(k)}-x^{(k-1)}||\ge err$}

\begin{algorithm}[htbp]
\caption{Gauss-Newton algorithm}
\begin{algorithmic}[1]
\Require{迭代初值$\Vbeta$,m个函数$\Vr=(r_1,r_2,\cdots,r_m)\myt$,误差限err}
\Ensure{$\Vbeta$的非线性最小二乘估计$\hat\Vbeta$。}
\State $k=0$
\State $\MJ=jacobian(\Vr,\Vbeta),\Vr=\Vr(\Vbeta)$
\While{$||\MJ\myt\Vr||>err $}
\State $\Vd=-(\MJ\myt\MJ)^{-1}\MJ\myt\Vr$
\State $\Vbeta=\Vbeta+\Vd$
\State $\MJ=jacobian(\Vr,\Vbeta),\Vr=\Vr(\Vbeta)$
\State $k=k+1$
\EndWhile
\State $\hat\Vbeta=\Vbeta$
\end{algorithmic}
\end{algorithm}

我们先来解释一下这个算法,给定$m$个关于n个变量$\Vbeta=(\beta_1,\beta_2,\cdots,\beta_n)\myt$的函数$\Vr=(r_1,\cdots,r_m)\myt$(也称为残差),$m\ge n$,Gauss-Newton算法找出了$\Vbeta$的值,使得平方和最小:
\begin{equation}
S(\Vbeta)=\frac{1}{2}\sum_{i=1}^mr_i^2(\Vbeta)=\frac{1}{2}\Vr(\Vbeta)\myt\Vr(\Vbeta)
\end{equation}
即求
\begin{equation}
\hat\Vbeta=\mathop{\arg\min}_{\Vbeta}S(\Vbeta)
\end{equation}
对$S(\Vbeta)$求导有:
\begin{equation}
\Vg=\nabla S(\Vbeta)=S'(\Vbeta)=\MJ(\Vbeta)\myt\Vr(\Vbeta)=0
\end{equation}
即驻点条件。

我们还有
\begin{equation}
\MH=S''(\Vbeta)=\MJ(\Vbeta)\myt\MJ(\Vbeta)+\sum_{i=1}^nr_i(\Vbeta)r''_i(\Vbeta)
\end{equation}

如果我们计算非线性方程$S'(\Vbeta)=\Vzero$的根,而已有近似值$\Vbeta^{(k)}$,采用牛顿法,我们将要计算
\begin{equation}
\Vbeta^{(k+1)}=\Vbeta^{(k)}-[\MH(\Vbeta^{(k)})]^{-1}\Vg(\Vbeta^{(k)})
\end{equation}

由于矩阵$\MJ(\Vbeta)\myt\MJ(\Vbeta)$至少是半正定的,如果忽略$S''(\Vbeta)$的后面一项,例如$\Vr(\Vbeta)$接近线性的情况,我们就得到了Gauss-Newton法。

这个迭代开始于近似值$\Vbeta^{(0)}$这个方法继续进行通过迭代
\begin{equation}
\Vbeta^{(s+1)}=\Vbeta^{(s)}-(\MJ\myt\MJ)^{-1}\MJ\myt\Vr(\Vbeta^{(s)})
\end{equation}
这里,如果$\Vr$和$\Vbeta$是列向量,Jacobian矩阵\footnote{Jacobian矩阵$\MJ$也记为$\frac{\partial \Vr}{\partial \Vbeta}$}的元素定义为
\begin{equation}
(\MJ)_{ij}=\frac{\partial r_i(\Vbeta^{(s)})}{\partial\Vbeta_j}
\end{equation}
如果$m=n$,这个迭代简化为
\begin{equation}
\Vbeta^{(s+1)}=\Vbeta^{(s)}-\MJ^{-1}\Vr(\Vbeta^{(s)})
\end{equation}
这是Newton法在一维情况的直接推广。

在数据拟合中,目标是找出在模型函数$y=f(x,\Vbeta)$的参数$\Vbeta$让它能最好拟合一些数据点$(x_i,y_i)$,函数$r_i$即为残差。
\begin{equation}
r_i(\Vbeta)=y_i-f(x_i,\Vbeta)
\end{equation}

值得注意的是,$(\MJ\myt\MJ)^{-1}\MJ\myt$是$\MJ$的伪逆。

下面讲述相对定向的过程。

首先,我们定义三个旋转矩阵
\begin{equation}
\MR_x(\theta)=\begin{bmatrix}
1 & 0 & 0 \\
0 & \cos\theta & -\sin\theta \\
0 & \sin\theta & \cos\theta
\end{bmatrix},\quad \MR_y(\theta)
\begin{bmatrix}
\cos\theta & 0 & -\sin\theta \\
0 & 1 & 0 \\
\sin\theta & 0 & \cos\theta
\end{bmatrix},\quad \MR_z(\theta)
\begin{bmatrix}
\cos\theta & -\sin\theta & 0 \\
\sin\theta & \cos\theta & 0 \\
0 & 0 & 1
\end{bmatrix}
\label{eq:rotationmatrix}
\end{equation}

由于单独相对相对定向采用以$y$为主轴的转角系统,左片的旋转矩阵和右片的旋转矩阵分别为
\begin{align}
\MR_1 &=\MR_y(\phi_1)\MR_z(\kappa_1) \\
\MR_2 &=\MR_y(\phi_2)\MR_x(\omega_2)\MR_z(\kappa_2)
\end{align}
设$p_1(x_1,y_1)$的同名点为$p_2(x_2,y_2)$,n个同名点横向堆叠形成的矩阵为$img_1,img_2$,将其写成像空间坐标系中的坐标
\begin{align}
img_1=[img_1;-f\cdot \ones(1,n)]\\
img_2=[img_2;-f\cdot \ones(1,n)]
\end{align}

将其变换到像空间辅助坐标系中
\begin{align}
img_1=\MR_1\cdot img_1\\
img_2=\MR_2\cdot img_2\label{eq:img2}
\end{align}

由于摄影基线$\Vb$与两幅构象光线$\Vl_1,\Vl_2$共面
\begin{equation}
\Vb\cdot(\Vl_1\times\Vl_2)=
\begin{vmatrix}
b & 0 & 0 \\
X_1 & Y_1 & Z_1 \\
X_2 & Y_2 & Z_2
\end{vmatrix}=b(Y_1Z_2-Y_2Z_1)=0
\end{equation}

我们采用matlab中的点乘记号$\cdot\times$表示对应元素相乘,点幂记号$\cdot '$表示实数转置$^T$\footnote{在matlab中$'$表示共轭转置$^H$}我们就可以得到残差向量
\begin{equation}
\Vr=(img_1(2,:)\cdot\times img_2(3,:)-img_2(2,:)\cdot\times img_1(3,:))\cdot '
\end{equation}

将五个转角参数$\Vbeta=(\phi_1,\kappa_1,\phi_2,\omega_2,\kappa_2)\myt$的初值置为$\Vzero$,置$err=\num{1e-7}$,即可用Gauss-Newton法迭代求解。

\subsubsection{前方交会建立模型}

运用上一小节中求得的独立像对相对定向参数以及\ref{sub:tongming}中量测的同名点坐标,根据前方交会计算式计算模型坐标,得到像片控制点、待定点的模型坐标。

具体过程如下
\begin{enumerate}
\item 根据内定向参数将\ref{sub:tongming}中量测的同名点的像素坐标转换为像点(框标坐标系)坐标$(x,y)$。
\item 确定摄影基线$b$值。摄影基线$b=l\times(1-p)\times M,B=(b,0,0)$,式中$l$为航片沿航向的像片边长,$p$为航向重叠度,$M$为摄影比例尺分母。

在本实验中,$l=46.08\times 2\si{mm},\; p=0.6,\; M=(7680\times 0.2)\si{m}/(46.08\times 2)\si{mm}$,其中,$0.2$是像元宽度。
\item 利用上一小节中求得的相对定向元素以及确定的$b$值,逐点计算点投影系数以及每对同名点的模型坐标,为检验相对定向的准确性和测量同名像点坐标的准确性,计算各点的上下视差$q$,$q$值的大小与正负可用于精度检查。
\end{enumerate}

以下是具体的计算式。

\begin{enumerate}
\item 与上小节类似地,进行由式\ref{eq:rotationmatrix}到\ref{eq:img2}的过程,将同名像点的坐标转换到像空间辅助坐标系中得$img_1,img_2$。
\item 对其中的每一对同名像点计算点投影系数
\begin{align}
N_1=& \frac{bZ_2}{X_1Z_2-X_2Z_1} \\
N_2=& \frac{bZ_1}{X_1Z_2-X_2Z_1}
\end{align}
\item 计算模型点坐标及上下视差
\begin{align}
x=& N_1X_1 \\
y=& 0.5(N_1Y_1+N_2Y_2) \\
z=& N_1Z_1 \\
q=& N_1Y_1-N_2Y_2
\end{align}
\end{enumerate}

\begin{note}
如果遇到上下视差特别大的情况,应该将该点剔除,重新相对定向。
\end{note}

\subsubsection{模型绝对定向编程}

经过相对定向之后,我们已经得到了以相对定向选定的像空间辅助坐标系为基准的模型,而我们需要的是地面测量坐标。绝对定向就是完成这两个坐标系之间的转化。

利用前方交会所求的像片控制点的模型坐标以及\ref{sub:waiye}中外业测量所测得的像控点地面坐标进行绝对定向,求出模型的7个绝对定向参数---三个平移参数$\Delta X,\Delta Y,\Delta Z$,一个缩放参数$\lambda$,三个旋转参数$\phi,\omega,\kappa$。再利用所求得的绝对定向参数将地物点的模型坐标转换到地面坐标系统。

\paragraph{绝对定向参数的求取}一个相对的两张相片共有十二个外方位元素,像对定向求得了五个元素以后,要恢复像对的绝对位置,还要解求七个绝对定向元素。它需要地面控制点(至少两个平高点和一个高程点)来解求,这种变换,在数学上为一个不同原点的三维空间相似变换,采用的数学模型是三维线性相似变换式:
\begin{equation}
\begin{bmatrix}
X \\ Y \\ Z
\end{bmatrix}=
\lambda\MR\begin{bmatrix}
x \\ y \\ z
\end{bmatrix}+
\begin{bmatrix}
dx \\
dy \\
dz
\end{bmatrix}
\label{eq:similar}
\end{equation}
其中,$(X,Y,Z)\myt$为地面点坐标,$(x,y,z)\myt$为模型点坐标,$\MR=R_y(\phi)R_x(\omega)R_z(\kappa)$。

因为这个模型又是关于绝对定向七参数的非线性模型,故又可以使用Gauss-Newton法来求解。

由于地面测量坐标系是左手系,像空间辅助坐标系是右手系,由解析几何的理论可知,平移旋转是第一类正交变换\footnote{变换矩阵$\MA$的行列式$\det(A)=1$},反射是第二类正交变换\footnote{变换矩阵$\MA$的行列式$\det(A)=-1$},这两类变换是无法相互转换的,即由上述七参数模型不可能直接将地面测量坐标和模型坐标对应起来,所以需要通过反射变换将地面测量坐标$(N,E,U)$\footnote{实验指导书中将地面测量坐标也记作$(X,Y,Z)$,本报告中统一记为$(N,E,U)$}变到地面摄测坐标$(X,Y,Z)$,即
\begin{equation}
\begin{bmatrix}
X \\  Y \\ Z
\end{bmatrix}=
\begin{bmatrix}
0 & 1 & 0 \\
1 & 0 & 0 \\
0 & 0 & 1
\end{bmatrix}
\begin{bmatrix}
N \\ E \\ U
\end{bmatrix}
\end{equation}

直观地理解就是将$N,E$互换位置。如表\ref{tab:coor}
\begin{table}[htbp]
\centering
\caption{坐标转换表(单位:$\si{m}$)}
\label{tab:coor}
\begin{tabular}{|c|c|c|c|c|c|}
\hline
\multicolumn{3}{|c|}{地面测量坐标} & \multicolumn{3}{c|}{地面摄测坐标} \\ \hline
      N  &   E    &   U   &   X    &   Y    &   Z   \\ \hline
   5299.483     &   3065.544    &   3.658   &  3065.544     &  5299.483     & 3.658   \\
\hline
\end{tabular}
\end{table}

解算步骤如下:
\begin{enumerate}
\item 将控制点的模型坐标和地面坐标\footnote{指地面摄测坐标}重心化。设控制点的模型坐标矩阵\footnote{坐标为行向量,按列堆叠,以后不再说明}为$\MM_{n\times 3}$(model),地面坐标矩阵为$\MG_{n\times 3}$(ground)。则重心化步骤如下
\begin{align}
\MM=& \MM-\ones(n,3)\cdot \diag(\mean(\MM)) \\
\MG=& \MG-\ones(n,3)\cdot \diag(\mean(\MG))
\end{align}
\item 将变换公式“拉直”\footnote{下述公式中的$\myvec$表示按列的方向对矩阵拉直}形成Gauss-Newton法的迭代格式
\begin{align*}
\MR=& \MR_y(\phi)\MR_x(\omega)\MR_z(\kappa) \\
\Vg=& \myvec(\MG\myt) \\
\Vr=& \lambda\MR\MM\myt+\diag([dx,dy,dz])\cdot\ones(3,n) \\
\Vr=& \myvec(\Vr)
\end{align*}
对于上述公式可以这样理解,右乘矩阵是对列进行操作,左乘矩阵是对行进行操作,将对矩阵的变换看做是对矩阵每一列的变换。
\item 设定迭代的初值:$\phi=0,\omega=0,\kappa=0,\lambda=1,dx=0,dy=0,dz=0$
\item 用高斯牛顿法迭代求解,并求出残差向量$\Vr$,将残差超限的点剔除。重新相对定向。
\end{enumerate}

\paragraph{模型坐标转换}

对相对定向建立的模型进行缩放、平移、旋转,将模型纳入到地面坐标系,利用上面求得的7个参数以及三维线性相似变换式\ref{eq:similar},将前方交会得到的模型坐标转换到地面坐标。

步骤:
\begin{enumerate}
\item 根据三个旋转参数计算旋转矩阵$\MR=\MR_y(\phi)\MR_x(\omega)\MR_z(\kappa)$,和一个缩放参数$\lambda$,三个平移参数$dx,dy,dz$组成三维线性相似变式。
\item 将模型点坐标带入三维线性相似变换式计算模型点的地面摄测坐标。
\end{enumerate}

\subsection{CAD成图练习}

根据地物点在左右影像上的像素位置(\ref{sub:tongming}量测得到),利用\ref{sub:biancheng}中求得的内定向、相对定向、绝对定向参数,以及内定向、前方交会、三维线性相似变换程序,确定地物点的地面位置。利用AutoCAD程序对要求成图范围内的地物进行成图。

步骤:
\begin{enumerate}
\item 将像对上量取地物特征点的同名像素坐标,通过内定向转换为像框坐标系的坐标。
\item 利用\ref{sub:biancheng}中求得的相对定向元素,进行前方交会计算,得到地物特征点的模型坐标。
\item 利用\ref{sub:biancheng}中求得的绝对定向元素,对求得的地物特征点模型坐标进行三维线性相似变换,得到地面特征点的地面坐标。
\item 根据地物特征点的地面坐标、以及\ref{sub:tongming}中量测时所绘的示意草图,利用CAD生成所要求区域的地形图。
\end{enumerate}

% (如图\ref{fig:dixing})
% \begin{figure}[htbp]
% \centering
% \includegraphics[width=\textwidth]{tu.PNG}
% \caption{CAD地形图示例}
% \label{fig:dixing}
% \end{figure}

% \subsection{航片的数字微分纠正}

% 利用已有的遥感软件(如ENVI等),根据\ref{sub:waiye}中量测的像控点坐标进行同济校园影像的数字微分纠正。

% 步骤:
% \begin{enumerate}
% \item 纠正函数:多项式;
% \item 在待纠正影像和已纠正影像上寻找易于测量的同名像点,并量测出其像点坐标。
% \item 利用软件对所量测出的点进行平差计算,去掉Q值大的点,最后确定21个点作为纠正用的控制点。
% \item 利用软件进行航片的数字微分纠正。
% \end{enumerate}

\section{Kinect软件使用}

课程安排中,有一节课是让我们练习使用Kinect。

Kinect是微软在2010年6月14日对XBOX360体感周边外设正式发布的名字。它是一种3D体感摄影机(开发代号“Project Natal”),同时它导入了即时动态捕捉、影像辨识、麦克风输入、语音辨识、社群互动等功能。玩家可以通过这项技术在游戏中开车、与其他玩家互动、通过互联网与其他Xbox玩家分享图片和信息等。

在本次实习中,我们用Kinect进行了三维建模。首先,将Kinect开启,然后,一个人手持Kinect围绕着被摄景物缓慢走动,最后,将生成的文件保存,并用专门的软件打开并查看效果。

\begin{figure}[htbp]
\centering
\caption{Kinect三维建模}
\includegraphics[width=0.8\textwidth]{3d.PNG}
\label{fig:3d}
\end{figure}

如图\ref{fig:3d}所示,这是我们组三维建模的成果,可以看出,Kinect三维建模的效果还是算不错的。

之后我们将模型的长度和实际的长度进行了对比。先将电脑的宽度用卷尺量出,然后和电脑上点出的长度进行了对比,我记得大概30厘米误差在$3\sim 4\si{cm}$左右,说明其精度并不算高,这也可能与其用途为游戏有关。

