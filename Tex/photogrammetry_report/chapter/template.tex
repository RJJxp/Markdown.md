\chapter{个人实习报告模板}

每个人的个人实习报告是一章(chapter),章下面可分为节(section),小节(subsection),小小节(subsubsection),段(paragraph)等,下面分别呈现。

LateX不同于WYSIWYG(所见即所得,What You See Is What You Get)文字系统,是一种WYTIWYG(所想即所得)文字系统。

\section{第一节}

这里是第一节,每段之间空一行。\LaTeX{}程序会每段开头会自动缩进。

\subsection{第一小节}

这里是第一小节,常用的两个环境是enumrate和itemize,下面有示例。
\begin{enumerate}
\item 项目一。
\item 项目二。
\item 项目三。
\end{enumerate}

\begin{itemize}
\item 项目一。
\item 项目二。
\item 项目三。
\end{itemize}

\subsubsection{第一小小节}

这里是第一小小节

\paragraph{段标题} 如果需要,可以加段标题。

\section{第二节}

常用的强调说明有:\\
加粗:\textbf{粗体,boldface}\\
倾斜:\emph{强调,emphasize}\\
其余的事项一般不用手动调。

大家写好后不用编译(单独一章也无法编译,还需要加一些头),直接给我编译。如果需要使用奇技淫巧还是先编译一遍看看吧。

如果大家需要编译,建议下载最新版的\TeX{}Live,千万不要下载C\TeX。
