\documentclass[a4paper,16pt]{article}
    % 宏包的使用
    \usepackage{ctex}
    \usepackage{verbatim}
    \usepackage{graphicx}
    \usepackage{upgreek}
    \usepackage{longtable}
    \usepackage{geometry}
    \usepackage{graphics}
    \usepackage{booktabs}

\begin{document}
    \section{课堂感想}
    很感谢程老师请来业内的一些专家并开设这门课程,记得前后有李荣兴老师、岳昌智书记,上海测绘局出身的企业家,摄影测量的博士企业家,还有最后大名鼎鼎的南通测绘局局长。过早的课程内容已经记不大清了,倒是对后三位的印象极为深刻,也并从其中受到灵魂深处的触动。

    借此报告抒发一下自己的感想。
    
    如有冒犯,还请宽恕。
    
    \subsection{测绘局出身的企业家}

    这位企业家皮肤黑黝黝,一看就是测绘的业内人士。
    其自述在上海测绘局工作多年,已到达中高层。然而因为厌倦了体制内部,想去建立自己的事业,所以辞职,自己创业,经过多年打拼,才有了今天的三点测绘科技有限公司。
    
    “这就是我”,他指着PPT上的照片说道,“我在测绘局立了三年的水准尺”,言语中不知是豪情还是悲伤。照片的背景是上海通往崇明的桥,测绘局的工作人员在测水准,他在图片的左上方,看上去全身是汗。这让我想起了自己在井冈山测量的经历,那种大热天,满身黏糊糊。他晒得很黑,不知他的母亲见到此番场景会不会心疼她的儿子。

    他又讲起了自己创业的艰辛,我听着也觉得确实艰辛。创业初期入不敷出,共事的同事的孩子刚出生,居然还要家里倒贴钱,让我台下人也不禁抹了一把鼻涕。中途自己如何解决其他企业不能搞定的工程问题,申请高新企业,今日站在这里给我们做演讲。
真的不容易,白手起家,传统测绘。

    \subsection{摄影测量的博士企业家}
    人乐呵呵的,声音富有磁性,和胖乎乎的体格有点不符,倒显得更加沉稳、庄重。同样也是自己创业,可感觉他要轻松的多。李荣兴老师和武大院士联合培养的博士果然与众不同。
    
    他自己之所以开了自己的公司,据他自己的原话,大意为,有企业看上他博士论文中核心技术的生产能力,给其他人做不如自己做,把利润扣在自己的手里,于是说干就干,成功将自己的学术转化为商业价值,为上海这座城市的建设添砖加瓦。
    
    他的公司就在同济联合广场,离我们很近。
    
    我认为这才是同济大学学生应有的创业方式,更多的是科技上的创新与进步,而不是一头扎入传统测绘的深渊里,在苦海中浮沉。也让我对测绘学院的课程设置的重心产生了怀疑。


    \subsection{南通测绘院院长}
    经程老师推荐而来,听说南通测绘局有限公司在他的带领下,从一个落的破产企业到今天的员工成百。我以为会谈一些实质性的内容,所以充满期待。没想到不到十分钟,就幻想破灭。
    
    漫长的演讲中充满了官僚气味的鸡汤,啊,你要努力,啊,企业管理三要素,啊,成功的因素。莫名喜感,忍俊不禁。发了小册子,最后意味深长的合了影。
    
    一说到测绘局我就想到了自己在国家测绘局第一航测遥感院实习的经历,注意,是国家测绘局,而且是第一航测遥感院,就光名头来讲,应该是比南通测绘局要好一些。然而待遇如何?一个武大航测毕业的的硕士,工作三年,他亲口说月薪5500,不知道是不是真假。问一起工作的同事,他们笑着开玩笑说,别问我们工资,问了你以后肯定都不想来。虽然有住房补贴,但貌似可以忽略,有宿舍楼,但是以后有了女朋友要结婚,总不能把人家往宿舍楼里领吧,只能买房,买房工资一个月半平米都不够,想想也是心疼。
    
    所以我对所有测绘局都充满了警惕,到处卖情怀,讲名号,闭口绝不谈薪资的大多数都是骗子。
    
    然而南通局的这位领导,一直在介绍自己的公司里面,进行了哪些联谊活动和一些鸡汤。里面有一段印象即为深刻,是院长自己拿着喇叭鼓动员工积极参加公司活动,员工一脸死灰,毫无表情。
    
    我深深体会到了两个字。
    
    那便是官僚。



    \section{行业感想}
    大三快结束了,学院的课程也几乎上完了。可根本不知道自己学到了什么看家本领。无论是什么时代,在哪个国家,没有一技之长,便很难在社会上行走。
    
    从课程上各种企业家的演讲和自己的经历,对行业还是有些担忧,这担忧主要是针对传统的测绘行业和自己的学业能力。觉得自己有些废柴,除了编程入门能让我对未来生活稍有些信心外,其他无一例外让人绝望。
    
    上海测绘院企业家看的我真是心塞。这也映射了传统测绘行业的现状,想生存,一起来压低价格,所以导致苦力活泛滥。外加行业门槛较低,从业人员疯长。从985高校毕业,无论什么学历,如果想成就一番事业,毕业去测绘局工作的人,无特殊情况的话,都是脑子进水的。
    
    测绘局我是绝对不会去,虽然我也不清楚自己能在学术方面能做到什么样子,总体来说,对自己的实力没有什么较为真实的估计。所以只能走一步看一步。
    


\end{document}